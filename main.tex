\documentclass[compress]{beamer}
\usepackage{euscript,amsmath,amssymb,amsfonts,amsthm,epsfig,subfigure,color,graphicx,enumitem}
\usepackage[utf8]{inputenc}
\usepackage[french]{babel}
\usepackage[T1]{fontenc}
\setcounter{page}{1}
\usepackage{pgf,csquotes}
\usepackage{tikz}
\newsavebox{\mysavebox}
%\usepackage[autolinebreaks,useliterate]{mcode}
\usetikzlibrary{arrows,positioning}
\usepackage{letltxmacro}
\LetLtxMacro\olditemize\itemize
\LetLtxMacro\oldenumerate\enumerate

\usepackage[doi=false,
            isbn=false,
            url=false,
            bibstyle=authoryear,
            style=authoryear,
            backend=bibtex]{biblatex}
\AtEveryCitekey{%\clearfield{title}
    \clearfield{note}
    \clearfield{pages}
    \clearlist{location}
    \clearlist{publisher}
    \clearname{editor}}
\renewcommand*{\multicitedelim}{\\}

\addbibresource{bib.bib}
\addbibresource{references.bib}
\addbibresource{journals.bib}

\usepackage{beamerthemedefault, multimedia, wasysym, amssymb, kpfonts}

\useoutertheme{smoothbars}
\useinnertheme[shadow=true]{rounded}
\setbeamercovered{transparent}
\setbeamertemplate{navigation symbols}{}
\setbeamertemplate{footline}[frame number]
%\setbeamertemplate{itemize items}{$\multimapdotinv$}
%\setbeamertemplate{itemize items}{\textbf{$\strictfi$}}
\setbeamertemplate{enumerate items}[circle]
\setbeamertemplate{section in toc}[circle]
%\useoutertheme{infolines}
\definecolor{myColor}{HTML}{E9CFEC}
\setbeamercolor{block title}{bg=green}

\definecolor{violet}{rgb}{0.8, 0.6, 0.8}
\definecolor{green}{rgb}{0.8 ,.9, 0.8}
\definecolor{blue}{rgb}{0.3, 0 ,0.6}

\graphicspath{{./}{../}}

\newcommand{\includesound}[1]{
\movie[]%
{\structure{\includegraphics[keepaspectratio,width=.5cm]{figures/play}}}{#1}}

% footnote without numbers
% \let\oldfootnote\footnote
% \renewcommand\footnote[1]{\let\thefootnote\relax%
% \oldfootnote{#1}}

% title, url, authors, extensions
\newcommand\citenote[4]{\footnote{#3 \href{#2}{\structure{#1}} #4}}

%\usecolortheme[named=purple]{structure}

\title[]{\LARGE \bf Soutenance d'habilitation à diriger les recherches}

\subtitle{\og Modélisation Long-Terme de Signaux Sonores \fg}

\author{Mathieu Lagrange
% \includesound{sounds/roxanne_thePolice.wav} \citenote{http://www.irccyn.ec-nantes.fr/\~{}lagrange}{http://www.irccyn.ec-nantes.fr/~lagrange}{}{}
}

\institute[Irccyn]
{
\includegraphics[height=2cm]{figures/logoLs2n}\\
% \includegraphics[height=1.5cm]{figures/logoEcn}
% \includegraphics[height=1.5cm]{figures/logoCnrs}
}

\date[]{14 Novembre 2019}

\logo{\includegraphics[height=.6cm]{figures/logoCnrs}}
\setlist{label=\textrm{--}}
\begin{document}
% Update itemize to have a default overlay
%\renewcommand{\itemize}[1][<+(1)->]{\olditemize[#1]}
%\renewcommand{\enumerate}[1][<+(1)->]{\oldenumerate[#1]}

%\includeonlyframes{current}

\frame{\titlepage \thispagestyle{empty}}

% show section lists
\begin{frame}{Agenda} \tableofcontents \end{frame} % [pausesections]

\section{Curriculum Vit\ae}

\begin{frame}{Statut}
\begin{block}{Chargé de recherche CNRS classe normale}
\begin{description}
\item[2009] : recrutement par la commission interdisciplinaire (CID) 44 : \og Cognition, langage, traitement de l’information, systèmes \fg
\item[2012] : rattachement à la section 07 : \og Sciences de l'information \fg
\end{description}
\end{block}
\begin{block}{Qualifications CNU}
\begin{itemize}
\item section 27 : \og Informatique \fg
\item section 61 : \og Génie informatique, automatique et traitement du signal \fg
\end{itemize}
\end{block}
\end{frame}

\begin{frame}{Diplômes}
\begin{block}{Master Informatique (2001)}
\begin{description}
\item[Intitulé] : \og Accélération de la synthèse sonore \fg 
\item[Encadrement] : Sylvain Marchand 
\item[Lieu de soutenance] : Université de Bordeaux 1
\end{description}
\end{block}
\begin{block}{Doctorat Informatique (2004)}
\begin{description}
\item[Intitulé] : \og Modélisation long terme des signaux polyphoniques \fg 
\item[Direction] : Myriam Desainte-Catherine
\item[Encadrement] : Sylvain Marchand et Jean-Bernard Rault 
\item[Lieu de soutenance] : Université de Bordeaux 1 
\end{description}
\end{block}
\end{frame}


\begin{frame}{Carrière}
\small 
    \begin{tabular}{ll}
  2001-4 & {\bf Doctorant Université Bordeaux 1} \\
  &  Ingénieur de Recherche  à France Télécom R\&D Rennes \\
  & TECH/IRIS (équipe codage et multimédia) \\
  2004-5 & {\bf Enseignant chercheur (ATER)} au LaBRI (U. Bx. 1)  \\
  2005-6 & {\bf Enseignant chercheur (ATER)}  à l'Enseirb (U. Bx. 1) \\
  2006-7 & {\bf Post-doctorant} au sein du département d'informatique \\
  &  Université de Victoria, BC, Canada \\
 2007-8 &  {\bf Post-doctorant} au sein du département "Music Technology"  \\
  &  Université de McGill, QC, Canada \\
 2008-9 &  {\bf Post-doctorant} au sein de  l'équipe    \\
  & \og Acoustique Audio et Ondes \fg, Télécom ParisTech \\
 2009-13 &  {\bf Chercheur CNRS} au sein de l'équipe Analyse / Synthèse  \\
  & Ircam (Umr 9912), Paris \\
 2013- -- &  {\bf Chercheur CNRS} au sein de l'équipe \\
 & Signal, Images et Son (Sims)  \\
  & Ls2n (Umr 6004), Ecole Centrale de Nantes \\
\end{tabular}
\end{frame}


\begin{frame}{Contributions}

\begin{block}{Indices bibliométriques}
\begin{itemize}
\item 21 revues internationales à comité de lecture
\item 62 conférences internationales à comité de lecture
\item citations: 1784 (source Google Scholar, Oct. 2019)
\item indice h: 19 (source Google Scholar, Oct. 2019)
\end{itemize}
\end{block}
\begin{block}{Responsabilités}
\begin{itemize}
\item relecteur pour diverses revues et conférences du domaine 
\item adjoint à la direction de l'équipe SIMS (2018)
\item membre du comité directeur d'association sportive de l'\'Ecole Centrale Nantes
\end{itemize}
\end{block}
\end{frame}

\begin{frame}{Encadrement}
\begin{itemize}
  \item Rémi Foucard (2010 - 2013): \og Fusion multi-niveaux par boosting pour le tagging automatique \fg
  \item Grégoire Lafay (2013 - 2016): \og Simulation de scènes sonores environnementales : application à l'analyse sensorielle et à l'analyse automatique \fg
  \item Jean-Rémy Gloaguen (2015 - 2018): \og Estimation du niveau sonore de sources d'intérêt au sein de mélanges sonores urbains : application au trafic routier \fg
  \item Félix Gontier (2017 - --): \og Modélisation de signaux sonores par approches neuronales profondes \fg
  \item Tom Souaille (2019 - --): \og Conception interactive en design sonore \fg
\end{itemize}
\end{frame}

\section[Tâches]{Tâches en traitement du signal audio-numérique}


\begin{frame}{Traitement du signal audio-numérique}
\begin{block}{Besoins et tâches associées}
\begin{itemize}
\item Transmission : Codage
\item Indexation : Recherche d'Information (IR)
\item Création : Synthèse sonore
\end{itemize}
\end{block}
\begin{block}{Domaines d'application}
\begin{itemize}
\item Musique
\item Sons environnementaux
\end{itemize}
\end{block}
\end{frame}


\begin{frame}{Besoins de compacité}
\begin{block}{Verrou}
\begin{itemize}
\item une seconde de son : $$ x \in \mathbb{R}^{44100}$$
\item besoin d'une représentation $y$ plus compacte
\end{itemize}
\end{block}
\begin{block}{Types de compacité}
\begin{itemize}
\item Codage : compacité signal ($ y \in \mathbb{R}^{5000}$)
\item Recherche d'Information : compacité sémantique ($ y \in \mathbb{N}^{100}$)
\item Synthèse : compacité \alert{\og} \structure{signalo-sémantique} \alert{\fg} ($ y \in \mathbb{R}^{100}$)
\end{itemize}
\end{block}
\end{frame}

\section[CASA]{Analyse computationnelle de scènes auditives (CASA)}

\begin{frame}{Codage par transformée}

$$y = C(x) \in \mathbb{R}^Y | x \in \mathbb{R}^{X} \tilde{x} = C^{-1}(y), P_e(x) \simeq P_e(\tilde{x})$$
\begin{itemize}
\item $C$ : Quantification adaptative d'un équivalent de la Transformée de Fourier à Court Terme (TFCT)
\item $P_e$ : Modélisation de la sensibilité aux déformations de la membrane basilaire\citenote{Zwicker}{}{}{}
\end{itemize}
Gain : $Y<<X$
Validation : écoute
\end{frame}

\begin{frame}{Transformée de Fourier à Court Terme (TFCT)}
$$ X[m, t] = \sum_{n = - \infty}^{\infty} x[n] w[n-t] \mathrm{e}^{\frac{-2 \mathrm{j}  \pi m n}{N}} $$ 
\begin{center}
\includegraphics[width=.6\columnwidth]{figures/play} \\
\end{center}
\end{frame}

\begin{frame}{Compromis temps/fréquence}
\begin{center}
\includegraphics[width=.6\columnwidth]{figures/play}
\includegraphics[width=.6\columnwidth]{figures/play} \\
\end{center}
\end{frame}

\begin{frame}{Typologie des évènements sonores}
\begin{description}
  \item[parole] : sons voisés <a>, <o> / sons plosifs <pe>, <qe>;
  \item[communication animale] : hululement de chouettes / clics de localisation de chauve souris;
  \item[musique] : chant lyrique / percussions;
  \item[mécanique] : ventilation / marteau piqueur;
  \item[environnementaux] : vent faisant siffler des câbles / gouttes de pluie tombant sporadiquement.
\end{description}
\end{frame}

\begin{frame}{Compromis temps/fréquence}
\begin{center}
\includegraphics[width=.4\columnwidth]{figures/play} 
\includegraphics[width=.4\columnwidth]{figures/play} \\
\end{center}
	$\hookrightarrow{}$ mitiger ce compromis imposée par l'approche court-terme par l'utilisation d'\alert{\textit{a priori}} sur les sources d'intérêt.
\end{frame}

\begin{frame}{Analyse de Scènes Auditives (ASA)}
L'ASA\footfullcite{bregman1994auditory} étudie l'ensemble de traitements perceptifs permettant
\begin{itemize}
\item d'isoler les informations émanant de sources distinctes,
\item de les organiser en un tout cohérent.
\end{itemize}
\end{frame}


Exemple ASA modulations

Modele sinusoidal long terme

Analyse a court terme

Tracking lentement predictible et ne pas générer de hautes fréquences

critère de continuation un critère de formation de sources parmi d'autres

asa

criteres instantanés

critères 

casa

asa  (ref ellis)

normalized cuts

houle

constat

Ca ne marche pas, mais je peux dire pourquoi
Ca marche, mais je ne peux pas trop dire pourquoi

\section[Expérimentation]{Expérimentation en traitement du signal audio-numérique} 18

\section[Projet]{Projet de recherche} 10


% show subsection list
\begin{frame}{} \tableofcontents[currentsection] \end{frame}

\begin{frame}{Figure}
\begin{center}
\includegraphics[width=.6\columnwidth]{figures/play} \\
\end{center}
\end{frame}

\begin{frame}{Block}
\begin{block}{title}
\begin{itemize}
\item
\item
\item
\end{itemize}
\end{block}
\end{frame}


\begin{frame}{Itemize}
\begin{itemize}
\item
\item
\item
\end{itemize}
\end{frame}

\begin{frame}{Special}
\structure{Hey}: ..
\begin{itemize}
\item[+]
\item[--]
\end{itemize}
\citenote{title}{http://www.google.com}{author}{infos}
\end{frame}

\end{document}
