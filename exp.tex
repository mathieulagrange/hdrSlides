\begin{frame}{L'expérimentation}
\begin{center}
\includegraphics[width=.3\columnwidth]{figures/play} \\
\end{center}
\begin{description}
\item[$X$] entrée, signal, observation
\item[$Y$] sortie, prédiction
\item[$p$] processus, traitement, prédicteur
\item[$E$] protocole expérimental
\end{description}
\end{frame}

\begin{frame}{\alert{Tâche} : recherche d'information}
\begin{center}
\includegraphics[width=.3\columnwidth]{figures/play}
\end{center}
\begin{itemize}
\item $X$ en grande dimension
\item $L$ aussi finalement
\item but : $X_j=D(X_i)$, $L_j=d(L_i) \text{ssi} y_j=y_i$
\item propriétés désirées: invariance, stabilité aux déformations signal
\end{itemize}
\end{frame}

\begin{frame}{Communautés}
\begin{block}{Music Information Retrieval (MIR)}
\begin{itemize}
\item 2000 - --
\item Challenge: 18 tâches  (Mirex)
\item Conference : 100 articles (Ismir)
\end{itemize}
\end{block}
\begin{block}{Detection and Classification of
Acoustic Scenes and Events (DCASE)}
\begin{itemize}
\item 2013 - --
\item Challenge: 7 tâches
\item Workshop: 50 articles
\end{itemize}
\end{block}
\end{frame}

\begin{frame}{Challenges en IR}
\begin{center}
\includegraphics[width=.3\columnwidth]{figures/play} \\ % petit y
\end{center}
\begin{itemize}
\item Alternative à l'approche \og mon outil, mon jeu de données, ma métrique \fg
\item Cadre de référence aux biais assumés collectivement
\item Protocole expérimental \og clé en main \fg
\end{itemize}
\end{frame}

\begin{frame}{IR en audio}
\begin{center}
\includegraphics[width=.3\columnwidth]{figures/play}
\end{center}
\begin{description}
\item[$R$] TFCT à plusieurs résolutions, réseaux convolutionnels profonds
\item[$C$] ensembles de réseaux neuronaux profonds 
\end{description}
\end{frame}

\begin{frame}{Challenges en crise ?}
\begin{center}
\og Faites quelque chose d'intéressant !! \fg \\ 
\og ... de scientifique !! \fg
\end{center}
$\hookrightarrow{}$ placer l'effort sur un questionnement plutôt que sur la démonstration d'un outil
\end{frame}



\begin{frame}{Design de Challenge}
\begin{block}{Approche  \og psychologie expérimentale \fg}
\begin{itemize}
\item formulation d'une hypothèse : le degré de polyphonie impacte les algorithmes de détection d'évènement sonores
\item production de corpus avec un degré variable de polyphonie
\item choix d'un protocole expérimental adapté
\item analyse des résultats
\end{itemize}
\end{block} \footfullcitenomarkleft{lafayhal-01111381}
\end{frame}

X

E

reproducibilité

donoho

protocoles expérimentaux assez canonique

explanes

extension de bande

demonstration

y

cense
cnsmdp
%laborieux, mais enrichissant et gratifiant



\begin{frame}{Approches étudiées}
\begin{center}
\includegraphics[width=.3\columnwidth]{figures/play} \\
\end{center}
\begin{description}
\item[$X$] plus de contrôle : données simulées
\item[$E$] plus de formalisation : dévelopement d'expLanes
\item[$Y$] plus de maîtrise : collaboration avec les communautés expertes
\end{description}
\end{frame}

\begin{frame}{Itemize}
\begin{itemize}
\item
\item
\item
\end{itemize}
\end{frame}
