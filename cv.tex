\begin{frame}{Statut}
\begin{block}{Chargé de recherche CNRS classe normale}
\begin{description}
\item [2009] : recrutement par la commission interdisciplinaire (CID) 44 : \og Cognition, langage, traitement de l’information, systèmes \fg
\item [2012] : rattachement à la section 07 : \og Sciences de l'information \fg
\end{description}
\end{block}
\begin{block}{Qualifications CNU}
\begin{itemize}
\item section 27 : \og Informatique \fg
\item section 61 : \og Génie informatique, automatique et traitement du signal \fg
\end{itemize}
\end{block}
\end{frame}

\begin{frame}{Diplômes}
\begin{block}{Master Informatique (2001)}
\begin{description}
\item [Intitulé] : \og Accélération de la synthèse sonore \fg 
\item [Encadrement] : Sylvain Marchand 
\item [Lieu de soutenance] : Université de Bordeaux 1
\end{description}
\end{block}
\begin{block}{Doctorat Informatique (2004)}
\begin{description}
\item [Intitulé]: \og Modélisation long terme des signaux polyphoniques \fg 
\item [Direction]: Myriam Desainte-Catherine
\item [Encadrement]: Sylvain Marchand et Jean-Bernard Rault 
\item [Lieu de soutenance]: Université de Bordeaux 1 
\end{description}
\end{block}
\end{frame}


\begin{frame}{Carrière}
\small 
    \begin{tabular}{ll}
  2001-4 & {\bf Doctorant Université Bordeaux 1} \\
  &  Ingénieur de Recherche  à France Télécom R\&D Rennes \\
  & TECH/IRIS (équipe codage et multimédia) \\
  2004-5 & {\bf Enseignant chercheur (ATER)} au LaBRI (U. Bx. 1)  \\
  2005-6 & {\bf Enseignant chercheur (ATER)}  à l'Enseirb (U. Bx. 1) \\
  2006-7 & {\bf Post-doctorant} au sein du département d'informatique \\
  &  Université de Victoria, BC, Canada \\
 2007-8 &  {\bf Post-doctorant} au sein du département "Music Technology"  \\
  &  Université de McGill, QC, Canada \\
 2008-9 &  {\bf Post-doctorant} au sein de  l'équipe    \\
  & \og Acoustique Audio et Ondes \fg, Télécom ParisTech \\
 2009-13 &  {\bf Chercheur CNRS} au sein de l'équipe Analyse / Synthèse  \\
  & Ircam (Umr 9912), Paris \\
 2013- -- &  {\bf Chercheur CNRS} au sein de l'équipe \\
 & Signal, Images et Son (Sims)  \\
  & Ls2n (Umr 6004), Ecole Centrale de Nantes \\
\end{tabular}
\end{frame}


\begin{frame}{Contributions}

\begin{block}{Indices bibliométriques}
\begin{itemize}
\item 21 revues internationales à comité de lecture
\item 62 conférences internationales à comité de lecture
\item citations: 1784 (source Google Scholar, Oct. 2019)
\item indice h: 19 (source Google Scholar, Oct. 2019)
\end{itemize}
\end{block}
\begin{block}{Responsabilités}
\begin{itemize}
\item relecteur pour 8 revues et 12 conférences du domaine 
\item adjoint à la direction de l'équipe SIMS 
\item membre du comité directeur d'association sportive de l'\'Ecole Centrale Nantes
\end{itemize}
\end{block}
\end{frame}

\begin{frame}{Encadrement}
\begin{itemize}
  \item Rémi Foucard (2010 - 2013): \og Fusion multi-niveaux par boosting pour le tagging automatique \fg
  \item Grégoire Lafay (2013 - 2016): \og Simulation de scènes sonores environnementales : application à l'analyse sensorielle et à l'analyse automatique \fg
  \item Jean-Rémy Gloaguen (2015 - 2018): \og Estimation du niveau sonore de sources d'intérêt au sein de mélanges sonores urbains : application au trafic routier \fg
  \item Félix Gontier (2017 - --): \og Modélisation de signaux sonores par approches neuronales profondes \fg
  \item Tom Souaille (2019 - --): \og Conception interactive en design sonore \fg
\end{itemize}
\end{frame}